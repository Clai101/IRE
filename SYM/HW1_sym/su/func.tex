\section{Преобразование функций}

\subsection{Одномерные пердставления}

Здесь мне повезло тетраидер очень хорошая фигура и если выбирать любой 
базис то в нем все оси равновыделенны, это мы знаем еще с аналитической
механики, когда расчитывали моенты инерции тетраидера, поэтому не 
побоюсь этим воспользоваться.

Для начала очевидное представление $A_1$ так как оно единичное то 
а повороты происходят в различых осях то сответственно только $x^2+y^2+z^2$.

\subsection{Трехмерные передставления}

Явно я знаю только одно представление, это 4-мерное, при этом оно приводимо.
Выглядят преобразования следующим образом:
\begin{gather}
    D(1032)=
    \begin{pmatrix}
        0& 1& 0& 0\\
        1& 0& 0& 0\\
        0& 0& 0& 1\\
        0& 0& 1& 0\\
    \end{pmatrix}
    \in C_2, \
    D(1203)=
    \begin{pmatrix}
        0& 1& 0& 0\\
        0& 0& 1& 0\\
        1& 0& 0& 0\\
        0& 0& 0& 1\\
    \end{pmatrix}
    \in C_3, \
    D(0132)=
    \begin{pmatrix}
        1& 0& 0& 0\\
        0& 1& 0& 0\\
        0& 0& 0& 1\\
        0& 0& 1& 0\\
    \end{pmatrix}
    \in \sigma
\end{gather}
Можно по анологии построить остальные матрицы представления. Такие матрицы дейтвуют на пространстве
функций:
\begin{gather}
    \begin{pmatrix}
        0& 1& 0& 0\\
        0& 0& 1& 0\\
        1& 0& 0& 0\\
        0& 0& 0& 1\\
    \end{pmatrix}
    \begin{pmatrix}
        a\\ b\\ c\\ f
    \end{pmatrix}
    =
    \begin{pmatrix}
        a\\ b\\ c\\ f
    \end{pmatrix}
\end{gather}

Найдем такую матрицу что будет преобразовывть мои $D(g)$ в:

\begin{gather}
    S^{-1} D(g) S =
    \begin{pmatrix}
        A& C\\
        0& B
    \end{pmatrix};
    \ B = 3\times 3, \  A = 1 \times 1, \ C = 1 \times 3
\end{gather}

Тут позволю себе отойти от темы. Я говрил вам на паре что матрица
иваритного подпространства не всегда должна иметь $C = 0$. Явно это можно увидеть 
из следующих сображений.

\begin{gather}
    \begin{pmatrix}
        A& C\\
        0& B
    \end{pmatrix}
    \begin{pmatrix}
        x_1\\ x_2
    \end{pmatrix}
    =
    \begin{pmatrix}
        A x_1 + B x_2\\ B x_2
    \end{pmatrix}
\end{gather}

То есть $x_2$ ивариантное подпространство. Я счтаю то это тот самы случай когда
это важно, потому что если бы я искал $C = 0$ то боюсь что не нашел бы такого пердставления.

Возварыюсь к теме, иными словами я должен сотавит из функций $(a, b, c, f)$ 4
линейных комбинации котрые будут удовлетворять:

1. 3 из них переходят сами в себя при действии любого элента группы

2. 4 должны сотвлять полный базис в протранстве 4-х элментов

3. 3 из первого пункта при перестановке двух элементов любых 
элементов и циклической должны выржать ся друг через друга так что-бы
при не возникало линеных компбинвций этих любых двух функций.

1 - требование ивариантности трех мерного пространства.

2 - требование не вырожденности преобразований.

3 - требование позволит обозначить эти три линейные комбинации за $x, y, z$
и тогда при перобразованих будет происходить например следующее $B (x, y, z) = (\alpha z, \beta x, \gamma z)$.
Кароче говря мы найдем самый трех мерный базис функций. Но может возникнуть вопрос почему я говорю
только о одной циклической перестановке и престановке двух элементов, на самом деле этих двух действий 
достаточно чтобы потсроить любое другое действие группы.

Матрицу $S$ я угадал но надо сказть что она более менее тривильна:

\begin{gather}
    S = 
    \begin{pmatrix}
        1& -1& -1& 1\\
        1& 1& -1& -1\\
        1& -1& 1& -1\\
        1& 0& 0& 0
    \end{pmatrix}
\end{gather}

То-есть функции ы новом базисе
\begin{gather}
    \begin{pmatrix}
        a - b - c + f\\ a + b - c - f\\ a - b + c - f\\ a
    \end{pmatrix}
    ; \
    \begin{pmatrix}
        x\\ y\\ z
    \end{pmatrix}
    =
    \begin{pmatrix}
        a - b - c + f\\ a + b - c - f\\ a - b + c - f
    \end{pmatrix}
\end{gather}


Проверка 2-огг условия происходят явно, $det(S) \neq 0$. Проверку 3 условия я не знаю как делать в общем случае,
но ее можно сделать явно по перемножать матрици:

\begin{gather}
    S^{-1}D(1032)S = 
    \cfrac{1}{2} 
    \begin{pmatrix}
        2& 2& -2& -2\\
        0& 4& -2& -2\\
        0& 3& -3& -1\\
        0& 3& -1& -3
    \end{pmatrix}
\end{gather}
\begin{gather}
    S^{-1}D(1203)S = 
    \begin{pmatrix}
        1& 0& 0& 0\\
        0& 0& 1& 0\\
        0& 0& 0& 1\\
        0& 1& 0& 0
    \end{pmatrix}
\end{gather}
\begin{gather}
    S^{-1}D(0132)S =
    \cfrac{1}{2}
    \begin{pmatrix}
        2& -2& 2& -2\\
        0& -1& 3& -3\\
        0& -2& 4& -2\\
        0& -3& 3& -1
    \end{pmatrix}
\end{gather}

С остальными аналогично. Для 3 тоже можно проверит явно $D(1032) (a - b - c + f) = -(a - b + c - f) $ или 
$D(0132)(a - b + c - f) = (a - b - c + f)$ или $D(1203)(a - b + c - f) = (a + b - c - f)$ и так далее.

Дальше так же протсо можно понять, что трех мерное представление $T_2$ в трех мерном 
пространстве будет преобразовывать $(x, y, z)$ или для квадратичных функций
$(xy, xz, yz)$. Возможен вопрос почему не $T_1$ рассмтрим $\sigma$, обычно 
отражение преводит $z \to -z$ поэтому рассмтрев обратный элемент $T_2^{-1}(\sigma) z \to -z$.

\subsection{Двумерное предствление}

Пля двумерного представления все сложнее. Надо выбрать какойто ортогональный 
базис но прижтом сохранить равновыделенност каждой оси. Понятно что для линейных
функций такого базиса не сыщешь, а вот для квдратичных функций можно. Еще 
надо заметить две вещи, первая я немного соврал когда говорил что все оси
равно выделенны на самомделе это немного не так, если мы можем повренуть тетраидер
и так что одна грань будет лежать например в $(Ox, Oy)$ и тогда у нас $x, y$
будет выделенной плоскостью. Второе, что мне нужно подметить так это, очевидно
что вумерное представлени берется из факторизации трехмерного на два одмнорных
и однодвумерное. Поэтому надо выбрать такой базис котрый будет ортогонален
$x^2 + y^2 + z^2$. И такой я знаю например $(x^2 - y^2, z^2 - \cfrac{1}{2} (x^2 + y^2))$ 



