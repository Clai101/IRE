\section{Поиcк представлений}

\subsection{Классы сопряженности}

С помощью простого перебора программой я получил следующий классы:

\begin{eqnarray}
    0123\\
    2301, 3210, 1032\\
    1203, 2130, 0231, 3102, 1320, 3021, 2013, 0312\\
    0132, 1023, 0321, 0213, 2103, 3120\\
    1230, 3201, 2031, 1302, 3012, 2310
\end{eqnarray}

\subsection{Анализ результатов классов смежности}

Сначала перепишем классы в дркгом виде для простоты объяснения:

\begin{eqnarray}
    e\\e
    C^{01}_2, C^{03}_2, C^{02}_2\\
    C^{21}_3, C^{10}_3, C^{13}_3, C^{20}_3, C^{12}_3, C^{11}_3, C^{22}_3, C^{23}_3\\
    s^{13}, s^{01}, s^{12}, s^{23}, s^{03}, s^{02}\\
    C^{02}_2*s^{23}, C^{02}_2*s^{01}, C^{01}_2*s^{12}, C^{01}_2*s^{03}, C^{01}_2*s^{02}, C^{01}_2*s^{13}
\end{eqnarray}

Как мы видим из не привиалных классов у нас есть 2 Нормальные подгрууппы поворотов
на $\pi$ и $\cfrac{2\pi}{3}$ соттветственно. Так же есть подгуппа отражений
и произведение отражений и повротов на $\pi$.

Как мы знаем всегда есть одно тривиаольное одномерное преодставление.
Но есть одно знако переменно порожденной не однозначностью выбора опредлителя
при поворотах, я могу с уверенность сказсть это так как группа повротов 
тераидера являтся подгуппой симмтрий шара тоесть группы $SO(3)$:


\begin{tabular}[pos]{c|c|c|c|c|c}
            & $e$ & $C_2$   & $C_3$ & $\sigma$  & $\sigma*C_2$  \\ \hline
    $A_1$   & $1$ & $1$     & $1$   & $1$       & $1$           \\ \hline
    $A_2$   & $1$ & $1$     & $1$   & $-1$      & $-1$          
\end{tabular}

Дальше пользуясь соотношением 
\begin{equation}
    C = \sum_i s^2_i = 1 + 1 + s^2_3 + s^2_4 + s^2_5
\end{equation}

Можем нати что оставшиеся 3 предсталения имеют следующие размерности 
$s_3 = s_4 = 3, \ s_5 = 2$. Можно было и рантше воспользоваться этим 
соотношениеим и порлучить, что всего есть 2 одномерных представления.

\subsection*{3-х Мерные представления}

Трех мреные мы всегда хорошо занаем так как у нас группа симмтрий 
тетраидера составленная из повротов и отражений, то матричные представления 
это повроты вокруг некотрых осей  матри поврота всегда можно передставить 
в виде:

\begin{gather}
    \begin{pmatrix}
        \cos \phi & \sin \phi & 0 \\
        \sin \phi & \cos \phi & 0 \\
        0 & 0 & \pm 1 
    \end{pmatrix}
\end{gather}

Как и в пршлый раз два предсталения это следствие не однозначности выбора 
определителя.

Получим характеры $\pm 1 + 2 \cos \phi$

\begin{tabular}[pos]{c|c|c|c|c|c}
        & $e$   & $C_2$ & $C_3$ & $\sigma$  & $\sigma*C_2$  \\ \hline
$A_1$   & $1$   & $1$   & $1$   & $1$       & $1$           \\ \hline
$A_2$   & $1$   & $1$   & $1$   & $-1$      & $-1$          \\ \hline
$T_1$   & $3$   & $-1$  & $0$   & $-1$      & $1$           \\ \hline
$T_2$   & $3$   & $-1$  & $0$   & $1$       & $-1$          
\end{tabular}

Для первых двух классов расчет одинкаов там $+1$:
\begin{eqnarray}
    e: \ 1 + 2 \cos \phi \ \vline_{\phi = 0} = 1 + 2 = 3 \\
    C_2: \ 1 + 2 \cos \phi \ \vline_{\phi = \pi} = 1 - 2 = -1 \\
    C_3: \ 1 + 2 \cos \phi \ \vline_{\phi = \cfrac{2 \pi}{3}} = 1 - 2 \cdot \cfrac{1}{2} = 0
\end{eqnarray}

Для последнийх двух столбцов отличие в знаке перед единицей, как в одномерных представленийх
собственно это они и есть толко записанные как блок трех мерной матрици.


\begin{eqnarray*}
    \sigma: \pm 1 \mp 2 \cos \phi \ \vline_{\phi = 0} = \pm 1 \mp 2 = \mp 1 \\
    \sigma * C_2: \mp 1 \pm \cos \phi \ \vline_{\phi = 0} = \pm 1
\end{eqnarray*}

\subsection{Двумерное представлене и таблица харектеров}

Последнюю строку для двумерного представления найдем из 
соотношения ортгональности.

\begin{gather}
    \begin{pmatrix}
        1 & 1 & 3 & 3 & 2
    \end{pmatrix}
    \begin{pmatrix}
        1 \\ 1 \\ -1 \\ -1 \\ x
    \end{pmatrix}
    = 1 + 1 - 3 - 3 + 2x = 0
    \implies 
    x = 2
\end{gather}


\begin{gather}
    \begin{pmatrix}
        1 & 1 & 0 & 0 & x    
    \end{pmatrix}
    \begin{pmatrix}
        1 \\ 1 \\ -1 \\ -1 \\ 2
    \end{pmatrix}
    = 1 + 1 + 2x = 0
    \implies 
    x = -1
\end{gather}

и так далее. В итоге получим:

\begin{tabular}[pos]{c|c|c|c|c|c}
    & $e$   & $C_2$ & $C_3$ & $\sigma$  & $\sigma*C_2$  \\ \hline
$A_1$   & $1$   & $1$   & $1$   & $1$       & $1$           \\ \hline
$A_2$   & $1$   & $1$   & $1$   & $-1$      & $-1$          \\ \hline
$T_1$   & $3$   & $-1$  & $0$   & $-1$      & $1$           \\ \hline
$T_2$   & $3$   & $-1$  & $0$   & $1$       & $-1$          \\ \hline
$E$     & $2$   & $2$   & $-1$  & $0$       & $0$          
\end{tabular}


